\documentclass[a4paper,fontsize=16pt]{article}
\usepackage[utf8]{inputenc}
\textheight=22.5cm % высота текста
\textwidth=17cm % ширина текста
\oddsidemargin= 0pt % отступ от левого края
\topmargin=-1.5cm % отступ от верхнего края
\parindent=24pt % абзацный отступ
\parskip=0pt % интервал между абзацами
\usepackage[russian]{babel}
\usepackage{titlepic}
\usepackage{graphicx}
\usepackage{amssymb}
\usepackage{amsmath}
\usepackage{graphicx}
\usepackage{color,colortbl, xcolor}
\usepackage{pgf}
\renewcommand\contentsname{Содержание}

\begin{document}

\begin{figure}
\includegraphics[width=5cm]{MIPT_logo.png}
\end{figure}

\begin{center}
\normalsize Московский физико-технический институт \\(госудраственный университет)
\end{center}
\vspace{3cm}
\begin{center}

\textbf {\huge{Практическое руководство по дифференцированию}}

\vspace{0.5cm}

\textbf {\large{Методическое пособие по дисциплине\break <<Введение в математический анализ>>\break\break 1 курс}}

\vspace{0.5cm}

\Large{{\textbf {Маслов А.С.}}}

\end{center}

\vspace{\fill}
\begin{center}
	Долгопрудный \\2021
\end{center}

\newpage

\tableofcontents

\newpage

\section{Введение}

\large{В пособии рассматриваются методы решения задач на взятие производных. В начале рассматриваются производные элементарных функций, приводится необходимая теория для решения сложных задач. Далее рассматриваются примеры решения задач.} 

\newpage

\section{Производные элементарных функций}


Из школьного курса алгебры известно, что

\begin{enumerate}
\item $c^\prime = 0$.
\item $x^\prime = 1$.
\item $(\sqrt{x})^\prime = \frac{1}{2\sqrt{x}}$.
\end{enumerate}


Производная степенной функции:

$(x^n)^\prime = n \cdot x^{n-1}$.
\vspace{0.5cm}

Производные показательной и логарифмической функций:

\begin{enumerate}
\item $(a^x)^\prime = a^x \cdot \ln{a}$
\item $(e^x)^\prime = e^x$
\item $\log_{a}{x} = \frac{1}{x\ln{a}}$
\item $(\ln{x})^\prime = \frac{1}{x}$
\end{enumerate}


Производные тригонометрических функций:

\begin{enumerate}
\item $(\sin{x})^\prime = \cos{x}$
\item $(\cos{x})^\prime = -\sin{x}$
\item $(\tan{x})^\prime = \frac{1}{(\cos{x})^2}$
\item $(\cot{x})^\prime = \frac{1}{(\sin{x})^2}$
\end{enumerate}


Производные обратных тригонометрических функций:

\begin{enumerate}
\item $(\arcsin{x})^\prime = \frac{1}{\sqrt{1 - x^2}}$
\item $(\arccos{x})^\prime = \frac{-1}{\sqrt{1 - x^2}}$
\item $(\arctan{x})^\prime = \frac{1}{1 + x^2}$
\end{enumerate}


Производные гиперболических функций:

\begin{enumerate}
\item $(\sinh{x})^\prime = \cosh{x}$
\item $(\cosh{x})^\prime = \sinh{x}$
\end{enumerate}


Данного теоретического материала достаточно, чтобы приступить к решению практических задач.

\newpage

\section{Примеры решения задач}

\hfill \break
\fbox{\textbf{Задача 1}}

Найти производную $fx$

\textbf{Решение:}


\newpage

\section{Список используемой литературы}

\begin{enumerate}
\item А.Ю.Петрович. Лекции по математическому анализу. Ч.1. Введение в математический анализ --- М.: Печатный салон ШАНС, 2017. --- 274 с.
\item Кудрявцев Л.Д, Курс математического анализа. Т.1. --- М.: Высшая школа, 1981. --- 688 с.
\item Никольский С.М. Курс математического анализа. Т.1. --- 4-е изд. --- М.: Наука, 1990. --- 528 с.
\item Тер-Крикоров А.М., Шабунин М.И. Курс математического анализа. --- М.: Наука, 1988. --- 816 с.
\item Яковлев Г.Н. Лекции по математическому анализу. Ч.1. --- М.: Физматлит, 2001. --- 400 с.
\item Бесов О.В. Лекции по математическому анализу. --- М.: Физматлит, 2014. --- 480 с.
\item Иванов Г.Е. Лекции по математическому анализу. Ч.1. --- М.: МФТИ, 2011. --- 318 с.
\item Фихтенгольц Г.М, Курс дифференциального и интегрального исчисления. --- 7-у изд. --- М.: Наука, 1969. --- 608 с.
\item Кудрявцев Л.Д., Кутасов А.Д., Чехлов В.И., Шабунин М.И. Сборник заадч по математическому анализу. Предел, непрерывность, дифференцируемость /под ред.  Л.Д.Кудрявцева. --- 2-е изд. --- М.: Физматлит, 2003. --- 496 с.
\item Демидович Б.П. Сборник задач и упражнений по математическому анализу. --- 10-е изд. --- М.: Наука, 1990. --- 624 с.
\end{enumerate}

\end{document}